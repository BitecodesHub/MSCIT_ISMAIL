\documentclass[11pt]{article}
\title{Sets: definition, and some operations}
\begin{document}
\maketitle
A {\bf set} is a collection of distinct, well-defined objects.

There is no requirement for the objects to be of a similar nature. 

The {\bf well-defined} attribute gives rise to the notions of {\bf truth, proof and computability}.
  
Every precisely defined statement has an unambiguous truth value even if it is not known. This is the truth.

Successfully establishing it is proof.

Supplying precise descriptions constitutes computability.

We have {\bf static sets}, which are unchanging, as well as dynamic ones that can be modified with time using inserts and deletes.

A subset is also a set, such that its every element comes from within the set. 

If $A$ and $B$ are sets then $A$ is a subset of $B$ if $\{x\in A\}\Rightarrow \{x\in B\}$.

Subset containment is not commutative. It is antisymmetric.

The subset containment is transitive:\\
$((A\subseteq B) \wedge (B \subseteq C)) \Rightarrow (A\subseteq B \subseteq C)$.

The number of subsets of an $n$ element set is $2^n$. This was proved using mathematical induction on $n$ and also by a bijective mapping to the set of all binary strings of length $n$. Binary strings are strings over $0$ and $1$.

We defined Intersection, Union and Complement. These correspond to the logical operations of AND, OR and NOT. For the third one, we needed a notion of a {\bf universal set}.

We looked at the concept of {\bf Venn Diagrams}. We looked at the formula for computing the union of sets called the Principle of Inclusion-Exclusion in its general form.

\end{document}

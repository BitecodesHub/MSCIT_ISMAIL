\documentclass[11pt]{article}
\title{Mathematical Logic- Axioms/ Inference rules}
\begin{document}
\maketitle
We looked at a list of some basic axioms of propositional logic:
\begin{enumerate}
\item {\bf Identity rules:}
\begin{itemize}
\item $p\wedge T\equiv p$
\item $p\vee F\equiv p$
\end{itemize}
\item {\bf domination rules:}
\begin{itemize}
\item $p\wedge  F\equiv F$
\item $p\vee T\equiv T$
\end{itemize}
\item {\bf Idompetent rules:}
\begin{itemize}
\item $p\vee p\equiv p$
\item $p\wedge p\equiv p$
\end{itemize}
\item {\bf ouble Negation rule:} 
\begin{itemize}
\item $\neg \neg p\equiv p$
\end{itemize}
\item {\bf Commutative rules:}
\begin{itemize}
\item $p\wedge q\equiv q\wedge p$
\item $p\vee q\equiv q\vee p$
\end{itemize}
\item {\bf Associative rules:}
\begin{itemize}
\item $((p\vee q)\vee r))\equiv (p\vee(q\vee r))$
\item $((p\wedge q)\wedge r))\equiv (p\wedge(q\wedge r))$
\end{itemize}
\item {\bf Distributive rules:}
\begin{itemize}
\item $p\wedge (q\vee r)\equiv (p \wedge q)\vee (p\wedge r)$
\item $p\vee(q\wedge r)\equiv (p\vee q)\wedge (p\vee r)$
\end{itemize}
\item {\bf de Morgan's rules:}
\begin{itemize}
\item $\neg(p\vee q)\equiv (\neg p)\wedge (\neg q)$
\item $\neg(p\wedge q)\equiv (\neg p)\vee (\neg q)$
\end{itemize}
\item {\bf Absorption rules:}
\begin{itemize}
\item $p\wedge (p\vee q)\equiv p$
\item $p\vee (p\wedge q)\equiv p$
\end{itemize}
\item {\bf Negation laws:}
\begin{itemize}
\item $p\vee (\neg p)\equiv T$ (called {\bf tautology/validity})
\item $p\wedge (\neg p)\equiv F$(called {\bf contradiction})
\end{itemize}
\end{enumerate}

Propositional logic has several inference rules, which may be used to derive complex formula as theorems, from simpler ones.
For example:
\begin{enumerate}
\item From $p\vee q$, $\neg p$ we can infer $q$
\item From $p\oplus q$ we can infer $p\vee q$
\item From $p\oplus q$ we can infer $\neg(p\wedge q)$
\end{enumerate}

There are some more elementary {bf inference rules} but the topic wont be dealt with more elaborately. A logic is considered {\bf sound} if there is a set of axioms and inference rules, such that any conclusion drawn by starting with some axioms and applying inference rules, is semantically a tautology. A logic is considered complete if every tautology in the semantic sense, has a corresponding syntactic formula that can be derived starting at some axioms and applying a series of inference rules. It is known that {\bf propositional logic} has a {system of axioms and inference rules} under whicht eh logic is {\bf sound and complete}. We will not be covering this in detail in this course.

A formula that has no atisfying assignments is said to be a {\bf contradiction}.

A formula that has at least one satisfying assignment is said to be a {\bf satisfiable formula}.

A formula that is true under all assignments is called a {\bf tautology} oir a {\bf validity}.

Typically a valid formula or tautology is one that is true under all assignments. One simple rule for constructing valid formulae is to take any two formulae $\psi_1$ and $\psi_2$ and such that under all truth assignments in which $\psi_1$ is true, $\psi_2$ is also true. In this case we can conclude that $\psi_1\Rightarrow \psi_2$. We saw this concept from the semantic point of view, using truth tables, in the lecture.

Students should construct examples of these on their own, for practice. Students should also learn to get a closed form formulae for number of satisfying assignments for various formulae following a pattern of operators. 

The problem of deciding algorithmically whether a given formula has a satisfying assignment is believed to be computationally very hard. This problem falls under the category of {\bf NP Complete Problems}. This is a technical term that wont be elaborated precisely. In simple terms, it indicates a problem whose best algorithm is likely to run in exponential time. A {\bf naive but imprecise explanation} for this phenomenon is that the number of possible truth assignments that need to be checked before we know whether it is satisfiable can be as high as $2^n$ where $n$ is the number of propositional variables.

We also had a quick look at conversion of integers between different number systems. A quick example here:

Write 600 in ternary (base 3; that is using only 0,1,2). 
First step list the various powers of 3.
1,3,9,27,81,243,729. 
The last on the list exceeds our target, so ignore it. The next number is 243. 600 is more than two times 243 but less than three times.
Thus we mark 2 in the position corresponding to $243=3^5$. The residue after performing $600-(2\times243)=114$. The next smallr power of 3 below 243 is 81. 81 is less than 114 but two times 81 is more than 114. Thus we mark 1 for the position $81=3^4$. The residue is now $114-81=33$. The biggest power of 3 heree is 27. This $27<33<2\times 27$. Thus we mark 1 for $27=3^3$. The residue is $33-27=6$. This is less than 9 thus we mark 0 for the position corresponding to $9=3^2$. Residue is still 6 which is greater than 3 and is in fact two times 3. We thus mark 2 for $3=3^1$. Residue is $6-6=0$. We thus mark 0 for $1=3^0$. We get our {\bf ternary representation of  decimal 600 as:}
\[211020\]

Students should practice converting numbers between bases 2,3,7,8, 10 etc.
\end{document}

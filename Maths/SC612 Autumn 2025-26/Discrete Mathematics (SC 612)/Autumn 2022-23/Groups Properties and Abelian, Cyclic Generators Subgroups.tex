\documentclass[11pt]{article}
\newtheorem{theorem}{Theorem}
\newtheorem{definition}{Definition}
\title{Functions Composition \& Permutation Groups}
\begin{document}
\maketitle
\begin{theorem}
The group identity element is unique.
\end{theorem}
Proof (by contradiction)

Assume there are two identities $e_1, e_2$ with $e_1\neq e_2$. Then,
\[e_2=e_1*e_2=e_1\]

\begin{theorem}
Every element in a group has a unique inverse.
\end{theorem}
Proof (by contradiction)

Suppose $x$ has two distinct inverses $y$ and $z$. Then,
\[x*y=y*x=x*z=z*x=e\]
Thus,
\[x*y=z*x\]
Premultiplying both sides by $z$, and using associativity, we get
\[(z*x)*y=z*(z*x)\]
This simplifies to \[e*y=z*e\]
or \[y=z\]

\begin{theorem}
Group equations have unique solutions. That is $g_1*x=g_2$ has exactly one solution (not more, nor less) where $x$ is unknown and $g_1$ and $g_2$ are two known elements of the group (possibly equal).
\end{theorem}
Proof

left multiplying both sides by $g_1^{-1}$ and using associativity, we get, 
\[x=g_1^{-1}*g_2\]

\begin{theorem}

\end{theorem}
\end{document}

\documentclass[11pt]{article}
\newtheorem{theorem}{Theorem}
\newtheorem{definition}{Definition}
\title{Functions \& Groups}
\begin{document}
\maketitle
A {\bf function} from domain $D$ to codomain $C$ is a special type of relation. The extra condition is that each element of the domain appears exactly once as the first coordinate of an ordered pair.

The number of functions is $n^m$ where $|D|=m$ and $|C|=n$.

A standard notation for the set of all functions from domain $D$ to codomain $C$ is $C^D$.

A group is a special type of ternary relation on a se. It can also be viewed as a function from $G\times G\rightarrow G$.

The basic definition of a group has an underlying set $G$ together with a binary operaion $*$, and these must satisfy the following four axions:
\begin{enumerate}
\item {\bf Closure:} $\forall g_1, g_2\in G, g_1*g_2 \in G$.
\item {\bf Identity}: $\exists e\in G, \forall g \in G, g*e=e*g=g$.
\item {\bf Inverses}: $\forall g\in G, \exists g'| g*g'=g'*g=e$.
\item {\bf Associativity}: $\forall g_1, g_2, g_3 \in G, (g_1*g_2)*g_3=g_1\*(g_2*g_3)$.
\end{enumerate}

Extra condition for {\bf Abelian groups}: $\forall g_1, g_2 \in G, g_1*g_2=g_2*g_1$.

Exampls of groups:
\begin{enumerate}
\item All Boolean functions over $k$ variables, under the $\oplus$ operation.
\item Integers modulo 5 under addition
\item Integers under addition.
\end{enumerate}
All these are Abelian groups. We will see non-Abelian groups next lecture.

A finite group may be represented as a matrix of $|G|\times |G|$ dimensions. The rows and columns are labelled with the elements and the entry $(i,j)$ is $g_i*g_j$.
\end{document}
